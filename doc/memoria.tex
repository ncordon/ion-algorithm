\documentclass[a4paper,11pt]{article}
% Símbolo del euro
\usepackage[gen]{eurosym}
% Codificación
\usepackage[utf8]{inputenc}
% Idioma
\usepackage[spanish]{babel} % English language/hyphenation
\selectlanguage{spanish}
% Hay que pelearse con babel-spanish para el alineamiento del punto decimal
\decimalpoint
\usepackage{dcolumn}
\newcolumntype{d}[1]{D{.}{\esperiod}{#1}}
\makeatletter
\addto\shorthandsspanish{\let\esperiod\es@period@code}
\makeatother

\usepackage[usenames,dvipsnames]{color} % Coloring code

\usepackage{csvsimple}
\usepackage{adjustbox}
\newsavebox\ltmcbox

% Ecuaciones aumentadas
\everymath{\displaystyle}

% Para matrices
\usepackage{amsmath}

% Símbolos matemáticos
\usepackage{amssymb}
\let\oldemptyset\emptyset
\let\emptyset\varnothing

% Hipervínculos
\usepackage{url}

\usepackage[section]{placeins} % Para gráficas en su sección.
\usepackage{mathpazo} % Use the Palatino font
\usepackage[T1]{fontenc} % Required for accented characters
\newenvironment{allintypewriter}{\ttfamily}{\par}
\setlength{\parindent}{0pt}
\parskip=8pt
\linespread{1.05} % Change line spacing here, Palatino benefits from a slight increase by default


% Imágenes
\usepackage{graphicx}
\usepackage{float}
\usepackage{caption}
\usepackage{wrapfig} % Allows in-line images



% Márgenes
\usepackage{geometry}
 \geometry{
 a4paper,
 total={210mm,297mm},
 left=30mm,
 right=30mm,
 top=25mm,
 bottom=25mm,
 }


% Referencias
\usepackage{fncylab}
\labelformat{figure}{\textit{\figurename\space #1}}

\usepackage{hyperref}
\hypersetup{
  colorlinks   = true, %Colours links instead of ugly boxes
  urlcolor     = blue, %Colour for external hyperlinks
  linkcolor    = blue, %Colour of internal links
  citecolor   = red %Colour of citations
}


\makeatletter
\renewcommand{\@listI}{\itemsep=0pt} % Reduce the space between items in the itemize and enumerate environments and the bibliography
\newcommand{\imagent}[4]{
  \begin{wrapfigure}{#4}{0.7\textwidth}
    \begin{center}
    \includegraphics[width=0.7\textwidth]{#1}
    \end{center}
    \caption{#3}
    \label{#4}
  \end{wrapfigure}
}


\newcommand{\imagen}[4]{
  \begin{minipage}{\linewidth}
    \centering
    \includegraphics[width=#4\textwidth]{#1}
    \captionof{figure}{#2}
    \label{#3}
  \end{minipage} 
}

\newcommand{\imgn}[3]{
  \begin{minipage}{\linewidth}
    \centering
    \includegraphics[width=#3\textwidth]{#1}
    \captionof{figure}{#2}
  \end{minipage} 
}

% Ejemplo de parámetro: ILS.r
\newcommand{\hrefr}[1]{
\href{../bin/#1}{#1}
}

%Customize enumerate tag
\usepackage{enumitem}
%Sections don't get numbered
%\setcounter{secnumdepth}{0}
\newcommand{\blue}[1]{\textcolor{blue}{#1}}
\begin{document}
%%%%%%%%%%%%%%%%%%%%%%%%%%%%%%%%%%%%%%%%%
% University Assignment Title Page 
% LaTeX Template
% Version 1.0 (27/12/12)
%
% This template has been downloaded from:
% http://www.LaTeXTemplates.com
%
% Original author:
% WikiBooks (http://en.wikibooks.org/wiki/LaTeX/Title_Creation)
% Modified by: NCordon (https://github.com/NCordon)
%
% License:
% CC BY-NC-SA 3.0 (http://creativecommons.org/licenses/by-nc-sa/3.0/)
% 
% Instructions for using this template:
% This title page is capable of being compiled as is. This is not useful for 
% including it in another document. To do this, you have two options: 
%
% 1) Copy/paste everything between \begin{document} and \end{document} 
% starting at \begin{titlepage} and paste this into another LaTeX file where you 
% want your title page.
% OR
% 2) Remove everything outside the \begin{titlepage} and \end{titlepage} and 
% move this file to the same directory as the LaTeX file you wish to add it to. 
% Then add \input{./title_page_1.tex} to your LaTeX file where you want your
% title page.
%
%%%%%%%%%%%%%%%%%%%%%%%%%%%%%%%%%%%%%%%%%
\begin{titlepage}

\newcommand{\HRule}{\rule{\linewidth}{0.5mm}} % Defines a new command for the horizontal lines, change thickness here

\center % Center everything on the page
 
%----------------------------------------------------------------------------------------
%	HEADING SECTIONS
%----------------------------------------------------------------------------------------
\textsc{\LARGE Universidad de Granada}\\[1.5cm]
\textsc{\Large Metaheaurísticas}\\[0.5cm] 

%----------------------------------------------------------------------------------------
%	TITLE SECTION
%----------------------------------------------------------------------------------------
\bigskip
\HRule \\[0.4cm]
{ \huge \bfseries Práctica Opcional}\\[0.4cm] % Title of your document
{ \huge \bfseries Algoritmo de Real Coding}\\
\HRule \\[1.5cm]
 
%----------------------------------------------------------------------------------------
%	AUTHOR SECTION
%----------------------------------------------------------------------------------------

\begin{minipage}{\textwidth}
\begin{center} \large
\emph{Algoritmo de iones}\\
\end{center}
\end{minipage}

%----------------------------------------------------------------------------------------
%	LOGO SECTION
%----------------------------------------------------------------------------------------

\begin{center}
\includegraphics[width=8cm]{./ugr.jpg}
\end{center}
%----------------------------------------------------------------------------------------

\begin{minipage}{\textwidth}
\begin{center} \large
Ignacio Cordón Castillo, 25352973G\\
\url{nachocordon@correo.ugr.es}\\
\ \\
$4^{\circ}$ Doble Grado Matemáticas Informática\\
Grupo Prácticas Viernes
\end{center}
\end{minipage}


\vspace{\fill}% Fill the rest of the page with whitespace
\large\today
\end{titlepage}  

\newpage
\tableofcontents
\newpage
% Examples of inclussion of images
%\imagent{ugr.jpg}{Logo de prueba}{ugr}
%\imagen{ugr.jpg}{Logo de prueba}{ugr2}{size relative to the \textwidth}

\section{Introducción}
Como sabemos, se trata de mejorar una metaheurística para minimizar las 20 primeras funciones 
de codificación real de la competición CEC2014, usando:

\begin{itemize}
  \item Dimensión D valuada a 10 y 30.
  \item Número de evaluaciones de cada función en una ejecucion del algoritmo: 100000 en dimensión 10, 300000
  en dimensión 30.
  \item 25 ejecuciones por función.
  \item Espacio de búsqueda de soluciones: $[-100, 100]^D$
\end{itemize}

Se compararán las medias obtenidas con estas premisas de ejecución para cada función. Además, debemos tener
en cuenta que el mínimo de la función $i-$ésima es $100\cdot i$. Cuando obtengamos los resultados, computarermos
el error respecto a dicho mínimo que hemos obtenido, de manera que buscaremos que los resultados en media
para cada una de las funciones se aproximen lo máximo posible a 0.

\section{Estado del arte}

La mayoría de técnicas para resolver problemas de optimización de funciones (las que nos ocupan son no lineales y muchas no diferenciables)
son algoritmos poblaciones, un tipo de algoritmos estocásticos en los que tenemos una serie de soluciones a las que, aplicados una serie de
operadores, van convergiendo a una posible solución del problema. Este tipo de algoritmos pueden ser divididos en tres subcategorías:

\begin{itemize}
 \item \textit{Algoritmos bioinspirados}.
 
  Basados mayormente en el modelado del comportamiento de enjambres o manadas de animales
  en la naturaleza. En este sentido, destaca PSO, algoritmo que imita el movimiento de las aves, o ACO (\textit{Ant Colony
  Optimization}). Recientemente han surgido toda una serie de algoritmos bioinspirados: \textit{Bat Algorithm}, \textit{Firefly
  Algorithm}, \textit{Cuckoo Search}, \textit{Grey Wolf Optimizatier}
  
 \item \textit{Algoritmos genéticos}.
 
  Los primeros estudios versaban sobre algoritmos genéticos con codificación real donde los investigadores se centraron durante mucho tiempo en
  encontrar nuevos operadores de cruce fructíferos. Fueron Rechenberg y Schwefel quienes propusieron en 1964 las estrategias
  de evolución, que se basaban en la auto-adaptabilidad de los parámetros. Nikolaus Hansen obtuvo en 1996 muy buenos resultados
  a este respecto modelando una adaptación mediante matrices de covarianza a través de su algoritmo CMAES.
  
  Posteriormente surgió la evolución diferencial, propuesta por Storm en 1997. Se trataba de un proceso estocástico por el que
  llevando una población de tuplas $n$-dimensionales, se iban mutando y recombinando para dar lugar a nuevas soluciones en un espacio
  de búsqueda. A este respecto, tenemos por ejemplo SaDE o JADE, algoritmos de Differential Evolution que auto-adaptan los 
  parámetros a los espacios de búsqueda para generar soluciones. Históricamente, la potencia de los algoritmos de Differential
  Evolution ha sido altísimo. De hecho una variante de este algoritmo, L-SHADE, resultó vencedora del reto completo (30 funciones,
  no 20) al que nos enfrentamos.
  
  En este campo también sobresalen la \textit{Programación Genética} y el \textit{Biogeography-based Optimizer}

 \item \textit{Algoritmos inspirados en la física}: la inspiración de estos algoritmos surge de leyes o fenómenos físicos.
 La diferencia respecto a los algoritmos bioinspirados es que la recombinación de soluciones se modela siguiendo ciertas
 leyes físicas, o inspirándose en ellas. Podemos citar dentro de esta subcategoría:
 
  \begin{itemize}
    \item \textit{Algoritmo de Optimización Magnética}: simula una serie de fuerzas electromagnéticas para desplazar las posiciones
    de las posibles soluciones en el espacio de búsqueda, ejerciendo las mejores soluciones mayor atracción hacia ellas.
    
    \item \textit{Algoritmo de Búsqueda Gravitacional}: se consideran una serie de masas que se atraen, siguiendo la ley
    Newtoniana del movimiento, en base a fuerzas modeladas a partir del valor que cada masa (o posible solución) produce
    en la función de \textit{fitness}.
    
    \item \textit{Algoritmo de iones}: el algoritmo que nos ocupa. Se modelan las soluciones en el espacio a través de posciones
    que pueden tener carga positiva o negativa. Cargas de distinto signo se atraen, mientras que cargas de signo opuesto se repelen.
  \end{itemize}
\end{itemize}

Al respecto del algoritmo que hemos escogido, el \textit{algoritmo de iones} (IMO) sólo existe un trabajo publicado por Behzad Javidy, 
Abdolreza Hatamlou y Seyedali Mirjalili, publicado a 19 de Marzo de 2015. Los autores tratan de modelar de forma
simplificada el comportamiento de los iones en el espacio y lo comparan con algoritmos genéticos, con Differential Evolution,
con PSO y con algoritmos de Colonia de Hormigas sobre 10 funciones test,
 obteniendo mejores medias y mejores desviaciones
típicas que los citados algoritmos para unos parámetros prefijados.

\section{Descripción del algoritmo}

El algoritmo se basa en el movimiento de los iones(soluciones) en el espacio. Se inicia una población de iones (igual número
de cationes que de aniones) a soluciones aleatorias en el espacio.
Se modelan dos fases en el movimiento de los iones:
\begin{itemize}
 \item \textbf{Fase líquida}
 
 Para cada iteración en la que nos encontremos en fase líquida, efectuamos lo descrito a continuación.
 
 Se encuentra el mejor catión $C_{best}$ de entre la población, en términos de \textit{fitness}.
 
 Para el anión $i-$ésimo, $A_i$, y su componente $j-$ésima, se modela una fuerza de atracción:
 
 $$F_{i,j} = \frac{1}{1+e^{d_{i,j}}} \qquad \qquad d_{i,j} = {\frac{-0.1}{|A_{i}[j] - C_{best}[j]|}}$$ 
 
 y se actualiza la componente del anión a:
 
 $$A_i[j] = A_i[j] + F_{i,j} \cdot (C_{best}[j] - A_i[j])$$
 
 Se efectúa un procedimiento análogo con los cationes, acercándolos al mejor anión.
 
 La idea es correcta, dado que si tenemos un catión (resp.anión) bueno, se modifican los aniones para que las componentes se
 empiecen a parecer a los de ese catión, con mayor grado de convergencia en las componentes más cercanas, y menor en las más
 lejanas.
 
 Además, la convergencia a puntos en el espacio debería estar garantizada, puesto que la sucesión $\left\{\frac{1}{2^n}\right\}$ converge
 a 0, y se tiene $ \frac{1}{2} \le F_{i,j} \le 1$. Pasaremos a discutir esto en la siguiente sección.
 
 
 \item \textbf{Fase sólida}: Se modela una fase en la que una vez alcanzada la convergencia a una solución, podamos movernos
 a otros puntos del espacio.
 
 Efectuamos el siguiente proceso, que es una modificación del propuesto por los autores, puesto que en el proceso descrito
 en el artículo científico original se presentan algunas imprecisiones:
 
  \small{\texttt{% Generator: GNU source-highlight, by Lorenzo Bettini, http://www.gnu.org/software/src-highlite
\noindent
\mbox{}\textbf{\textcolor{Blue}{if}}\ \ fitness\textcolor{BrickRed}{(}best\textcolor{BrickRed}{(}cations\textcolor{BrickRed}{))}\ \textcolor{BrickRed}{\textgreater{}=}\ fitness\textcolor{BrickRed}{(}worst\textcolor{BrickRed}{(}cations\textcolor{BrickRed}{))}\ \textbf{\textcolor{Blue}{and}} \\
\mbox{}\ \ \ \ fitness\textcolor{BrickRed}{(}best\textcolor{BrickRed}{(}anions\textcolor{BrickRed}{))}\ \textcolor{BrickRed}{\textgreater{}=}\ fitness\textcolor{BrickRed}{(}worst\textcolor{BrickRed}{(}anions\textcolor{BrickRed}{))} \\
\mbox{} \\
\mbox{}\ \ \ \ \textbf{\textcolor{Blue}{if}}\ \textcolor{BrickRed}{(}rand\ \textcolor{BrickRed}{\textless{}}\ prob$\_$restart\textcolor{BrickRed}{)}\textcolor{Red}{\{} \\
\mbox{}\ \ \ \ \ \ \ \ random$\_$restart\textcolor{BrickRed}{(}anions\textcolor{BrickRed}{)} \\
\mbox{}\ \ \ \ \textcolor{Red}{\}} \\
\mbox{}\ \ \ \ \textbf{\textcolor{Blue}{else}}\textcolor{Red}{\{} \\
\mbox{}\ \ \ \ \ \ \ \ \textbf{\textcolor{Blue}{for}}\ i\ \textbf{\textcolor{Blue}{in}}\ \textcolor{Red}{\{}\textcolor{Purple}{1}\textcolor{BrickRed}{,...}length\textcolor{BrickRed}{(}anions\textcolor{BrickRed}{)}\textcolor{Red}{\}\{} \\
\mbox{}\ \ \ \ \ \ \ \ \ \ \ \ \textbf{\textcolor{Blue}{if}}\ \textcolor{BrickRed}{(}rand\textcolor{BrickRed}{(}\textcolor{Purple}{0}\textcolor{BrickRed}{,}\textcolor{Purple}{1}\textcolor{BrickRed}{)}\ \textcolor{BrickRed}{\textgreater{}}\ \textcolor{Purple}{0.5}\textcolor{BrickRed}{)} \\
\mbox{}\ \ \ \ \ \ \ \ \ \ \ \ \ \ \ \ anion\textcolor{BrickRed}{[}i\textcolor{BrickRed}{]}\ \textcolor{BrickRed}{=}\ anion\textcolor{BrickRed}{[}i\textcolor{BrickRed}{]}\ \textcolor{BrickRed}{+}\ rand\textcolor{BrickRed}{(-}\textcolor{Purple}{1}\textcolor{BrickRed}{,}\textcolor{Purple}{1}\textcolor{BrickRed}{)*(}best$\_$old$\_$cation\textcolor{BrickRed}{)} \\
\mbox{}\ \ \ \ \ \ \ \ \ \ \ \ \textbf{\textcolor{Blue}{else}} \\
\mbox{}\ \ \ \ \ \ \ \ \ \ \ \ \ \ \ \ anion\textcolor{BrickRed}{[}i\textcolor{BrickRed}{]}\ \textcolor{BrickRed}{=}\ anion\textcolor{BrickRed}{[}i\textcolor{BrickRed}{]}\ \textcolor{BrickRed}{+}\ rand\textcolor{BrickRed}{(-}\textcolor{Purple}{1}\textcolor{BrickRed}{,}\textcolor{Purple}{1}\textcolor{BrickRed}{)*(}best$\_$cation\textcolor{BrickRed}{)} \\
\mbox{}\ \ \ \ \ \ \ \ \textcolor{Red}{\}} \\
\mbox{}\ \ \ \ \textcolor{Red}{\}} \\
\mbox{}\ \ \ \ \textbf{\textcolor{Blue}{if}}\ \textcolor{BrickRed}{(}rand\ \textcolor{BrickRed}{\textless{}}\ prob$\_$restart\textcolor{BrickRed}{)}\textcolor{Red}{\{} \\
\mbox{}\ \ \ \ \ \ \ \ random$\_$restart\textcolor{BrickRed}{(}cations\textcolor{BrickRed}{)} \\
\mbox{}\ \ \ \ \textcolor{Red}{\}} \\
\mbox{}\ \ \ \ \textbf{\textcolor{Blue}{else}}\textcolor{Red}{\{} \\
\mbox{}\ \ \ \ \ \ \ \ \textbf{\textcolor{Blue}{for}}\ i\ \textbf{\textcolor{Blue}{in}}\ \textcolor{Red}{\{}\textcolor{Purple}{1}\textcolor{BrickRed}{,...}length\textcolor{BrickRed}{(}cations\textcolor{BrickRed}{)}\textcolor{Red}{\}\{} \\
\mbox{}\ \ \ \ \ \ \ \ \ \ \ \ \textbf{\textcolor{Blue}{if}}\ \textcolor{BrickRed}{(}rand\textcolor{BrickRed}{(}\textcolor{Purple}{0}\textcolor{BrickRed}{,}\textcolor{Purple}{1}\textcolor{BrickRed}{)}\ \textcolor{BrickRed}{\textgreater{}}\ \textcolor{Purple}{0.5}\textcolor{BrickRed}{)} \\
\mbox{}\ \ \ \ \ \ \ \ \ \ \ \ \ \ \ \ cation\textcolor{BrickRed}{[}i\textcolor{BrickRed}{]}\ \textcolor{BrickRed}{=}\ cation\textcolor{BrickRed}{[}i\textcolor{BrickRed}{]}\ \textcolor{BrickRed}{+}\ rand\textcolor{BrickRed}{(-}\textcolor{Purple}{1}\textcolor{BrickRed}{,}\textcolor{Purple}{1}\textcolor{BrickRed}{)*(}best$\_$old$\_$anion\textcolor{BrickRed}{)} \\
\mbox{}\ \ \ \ \ \ \ \ \ \ \ \ \textbf{\textcolor{Blue}{else}} \\
\mbox{}\ \ \ \ \ \ \ \ \ \ \ \ \ \ \ \ cation\textcolor{BrickRed}{[}i\textcolor{BrickRed}{]}\ \textcolor{BrickRed}{=}\ cation\textcolor{BrickRed}{[}i\textcolor{BrickRed}{]}\ \textcolor{BrickRed}{+}\ rand\textcolor{BrickRed}{(-}\textcolor{Purple}{1}\textcolor{BrickRed}{,}\textcolor{Purple}{1}\textcolor{BrickRed}{)*(}best$\_$anion\textcolor{BrickRed}{)} \\
\mbox{}\ \ \ \ \ \ \ \ \textcolor{Red}{\}} \\
\mbox{}\ \ \ \ \textcolor{Red}{\}} \\
\mbox{}
}}
  \normalsize

 La probabilidad de reinicio de soluciones propuesta por los autores en su \textit{paper} original es del 0.05, aunque hemos
 hecho un pequeño ajuste de los parámetros del algoritmo, a fin de optimizarlos.
 
 Cuando se reinician tanto los cationes como los aniones, se hace por supuesto a tuplas en el espacio de búsqueda: $[-100,100]^D$
 
 \texttt{best\_old\_cation} y \texttt{best\_old\_anion} representan los mejores iones de cada clase en la anterior ejecución del
 algoritmo, mientras \texttt{best\_cation} y \texttt{best\_anion} representan los mejores iones de cada clase en la actual 
 ejecución del algoritmo. Cuando se recombinan los actuales iones, cationes con los mejores cationes, iones respectivamente,
 el objetivo es explorar otras zonas del espacio de búsqueda. La eficacia de este aspecto será una de las cuestiones a debatir
 en las mejoras del algoritmo.
 
 Cuando no se reinician las soluciones, sino que se recombinan usando los mejores aniones y cationes de esta o anteriores ejecuciones,
 se observó que las soluciones no tenían porqué quedarse en el espacio de búsqueda, así que se implementó un sencillo procedimiento
 de normalización de soluciones, devolviéndolas a los límites de nuestro dominio:
 
  \small{\texttt{% Generator: GNU source-highlight, by Lorenzo Bettini, http://www.gnu.org/software/src-highlite
\noindent
\mbox{}\textbf{\textcolor{Blue}{for}}\ i\textcolor{BrickRed}{=}\textcolor{Purple}{1}\ to\ length\textcolor{BrickRed}{(}cations\textcolor{BrickRed}{)}\textcolor{Red}{\{} \\
\mbox{}\ \ \ \ \textbf{\textcolor{Blue}{for}}\ j\textcolor{BrickRed}{=}\textcolor{Purple}{1}\ to\ length\textcolor{BrickRed}{(}cation\textcolor{BrickRed}{[}i\textcolor{BrickRed}{])} \\
\mbox{}\ \ \ \ \ \ \ \ \textbf{\textcolor{Blue}{if}}\ \textcolor{BrickRed}{(}cation\textcolor{BrickRed}{[}i\textcolor{BrickRed}{][}j\textcolor{BrickRed}{]}\ \textcolor{BrickRed}{\textless{}}\ \textcolor{BrickRed}{-}\textcolor{Purple}{100}\textcolor{BrickRed}{)}\ cation\textcolor{BrickRed}{[}i\textcolor{BrickRed}{][}j\textcolor{BrickRed}{]}\ \textcolor{BrickRed}{=}\ \textcolor{BrickRed}{-}\textcolor{Purple}{100} \\
\mbox{}\ \ \ \ \ \ \ \ \textbf{\textcolor{Blue}{elsif}}\ \textcolor{BrickRed}{(}cation\textcolor{BrickRed}{[}i\textcolor{BrickRed}{][}j\textcolor{BrickRed}{]}\ \textcolor{BrickRed}{\textgreater{}}\ \textcolor{Purple}{100}\textcolor{BrickRed}{)}\ cation\textcolor{BrickRed}{[}i\textcolor{BrickRed}{][}j\textcolor{BrickRed}{]}\ \textcolor{BrickRed}{=}\ \textcolor{Purple}{100} \\
\mbox{}\textcolor{Red}{\}} \\
\mbox{} \\
\mbox{}\textbf{\textcolor{Blue}{for}}\ i\textcolor{BrickRed}{=}\textcolor{Purple}{1}\ to\ length\textcolor{BrickRed}{(}anions\textcolor{BrickRed}{)}\textcolor{Red}{\{} \\
\mbox{}\ \ \ \ \textbf{\textcolor{Blue}{for}}\ j\textcolor{BrickRed}{=}\textcolor{Purple}{1}\ to\ length\textcolor{BrickRed}{(}anion\textcolor{BrickRed}{[}i\textcolor{BrickRed}{])} \\
\mbox{}\ \ \ \ \ \ \ \ \textbf{\textcolor{Blue}{if}}\ \textcolor{BrickRed}{(}anion\textcolor{BrickRed}{[}i\textcolor{BrickRed}{][}j\textcolor{BrickRed}{]}\ \textcolor{BrickRed}{\textless{}}\ \textcolor{BrickRed}{-}\textcolor{Purple}{100}\textcolor{BrickRed}{)}\ anion\textcolor{BrickRed}{[}i\textcolor{BrickRed}{][}j\textcolor{BrickRed}{]}\ \textcolor{BrickRed}{=}\ \textcolor{BrickRed}{-}\textcolor{Purple}{100} \\
\mbox{}\ \ \ \ \ \ \ \ \textbf{\textcolor{Blue}{elsif}}\ \textcolor{BrickRed}{(}anion\textcolor{BrickRed}{[}i\textcolor{BrickRed}{][}j\textcolor{BrickRed}{]}\ \textcolor{BrickRed}{\textgreater{}}\ \textcolor{Purple}{100}\textcolor{BrickRed}{)}\ anion\textcolor{BrickRed}{[}i\textcolor{BrickRed}{][}j\textcolor{BrickRed}{]}\ \textcolor{BrickRed}{=}\ \textcolor{Purple}{100} \\
\mbox{}\textcolor{Red}{\}} \\
\mbox{}
}}
  \normalsize
 
 Asimismo, la condición que modelaban los autores originalmente para entrar en esta fase sólida era:
 
  \small{\texttt{% Generator: GNU source-highlight, by Lorenzo Bettini, http://www.gnu.org/software/src-highlite
\noindent
\mbox{}\ \ \ \ fitness\textcolor{BrickRed}{(}best\textcolor{BrickRed}{(}cations\textcolor{BrickRed}{))}\ \textcolor{BrickRed}{\textgreater{}=}\ fitness\textcolor{BrickRed}{(}worst\textcolor{BrickRed}{(}cations\textcolor{BrickRed}{))/}\textcolor{Purple}{2}\ \textbf{\textcolor{Blue}{and}} \\
\mbox{}\ \ \ \ fitness\textcolor{BrickRed}{(}best\textcolor{BrickRed}{(}anions\textcolor{BrickRed}{))}\ \textcolor{BrickRed}{\textgreater{}=}\ fitness\textcolor{BrickRed}{(}worst\textcolor{BrickRed}{(}anions\textcolor{BrickRed}{))/}\textcolor{Purple}{2} \\
\mbox{}
}}
  \normalsize
  
 Pero usar dicha división entre 2 parecía algo ilógico. Ésta constituye una de las grandes imprecisiones que se comentaban
 anteriormente. Por ello se suprimió dicha división por 2. Más aún, ni siquiera sin esa división por 2 parecemos asegurar que abandonemos
 la fase líquida en algún momento. Esta será una de las cuestiones a debatir en las mejoras del algoritmo.
\end{itemize} 
 
 \subsection{Equilibrio Diversidad Convergencia}
 
  Parece existir un problema con la convergencia a puntos del espacio, puesto que podríamos tener dos puntos del espacio que
  fueran óptimos locales en una cierta zona muy cercana, el mejor anión situado en uno de ellos, y el mejor catión en otro.
  Podríamos tener a todos los aniones excepto el mejor, en una zona lo suficientemente cercana al mejor catión; y a todos
  los cationes, excepto el mejor, en una zona lo suficientemente cercana al mejor anión. Se tendría que en cada iteración,
  todos los aniones se intercambian con los cationes, por la cercanía de las partículas, y todos los cationes se intercambian
  con los aniones, y nunca llegaríamos a tener la convergencia a un único punto.
  
  Matemáticamente, sea $f$ la función a minimizar en un espacio $[-100,100]^D$, y sean $x,y$ en dicho espacio. Supongamos:
  
  $$\exists B(x,\delta_1), B(y, \delta_2)$$ verificándose que $$B(x,\delta_1) \bigcap B(x,\delta_2) = \emptyset $$
  
  $f$ tiene un mínimo local en $B(x,\delta_1)$ y otro en $B(y,\delta_2)$
  
  se cumple que $C_{best}\in B(x,\delta_1)$, $A_{best}\in B(y,\delta_2)$, y además $A_i\in B(x,\delta_1) \forall A_i \neq A_{best}$ 
  y $C_i\in B(y,\delta_2) \forall C_i \neq C_{best}$, con $F_{i,j} \approx 1$. Podríamos entonces no tener convergencia
  nunca a un único punto, o incluso que fuese tan lenta que estuviésemos perdiendo un montón de evaluaciones con soluciones
  de \textit{fitness} similar.
  
  Es más, supongamos en lo que sigue que tenemos convergencia a un único punto. Aún así, el mecanismo para entrar
  a fase líquida no parece del todo certero, porque nunca podemos asegurar que se vaya  cumplir que:
  
  \small{\texttt{% Generator: GNU source-highlight, by Lorenzo Bettini, http://www.gnu.org/software/src-highlite
\noindent
\mbox{}\ \ \ \ fitness\textcolor{BrickRed}{(}best\textcolor{BrickRed}{(}cations\textcolor{BrickRed}{))}\ \textcolor{BrickRed}{\textgreater{}=}\ fitness\textcolor{BrickRed}{(}worst\textcolor{BrickRed}{(}cations\textcolor{BrickRed}{))/}\textcolor{Purple}{2}\ \textbf{\textcolor{Blue}{and}} \\
\mbox{}\ \ \ \ fitness\textcolor{BrickRed}{(}best\textcolor{BrickRed}{(}anions\textcolor{BrickRed}{))}\ \textcolor{BrickRed}{\textgreater{}=}\ fitness\textcolor{BrickRed}{(}worst\textcolor{BrickRed}{(}anions\textcolor{BrickRed}{))/}\textcolor{Purple}{2} \\
\mbox{}
}}
  \normalsize
  
  a pesar de haber alcanzado un óptimo local. Aún así, si este mecanismo fuese del todo certero, deberíamos buscar otro método
  para introducir mayor diversidad en la población de iones.
 
 \subsection{Comparación con las otras metaheaurísticas}
 
 Hemos obtenido los siguientes resultados, tras la implementación del algoritmo con las variaciones descritas, pero filedigno
 a lo expuesto por sus autores:
 
 
  \begin{table}[H]	
  \caption{Resultados del \textit{Algoritmo de iones}}
  \centering
  \begin{tabular}{|l|r|r|}
  \hline
  &  \textbf{Dimensión 10}& \textbf{Dimensión 30} \\ \hline
  f1 &  19084554.669857 &  78738913.351420 \\ \hline
  f2 &  16536.801547 &  13697676.211767 \\ \hline
  f3 &  21065.221623 &  77800.939852 \\ \hline
  f4 &  58.764875 &  246.343748 \\ \hline
  f5 &  20.170120 &  20.468675 \\ \hline
  f6 &  7.539682 &  34.790689 \\ \hline
  f7 &  0.861159 &  1.102618 \\ \hline
  f8 &  28.934028 &  129.837295 \\ \hline
  f9 &  26.123286 &  145.730468 \\ \hline
  f10 &  1157.285877 &  4219.909911 \\ \hline
  f11 &  1144.876264 &  4881.285910 \\ \hline
  f12 &  0.661762 &  1.304201 \\ \hline
  f13 &  0.456650 &  0.440387 \\ \hline
  f14 &  0.635537 &  0.260939 \\ \hline
  f15 &  4.164066 &  51.414376 \\ \hline
  f16 &  3.390798 &  12.366670 \\ \hline
  f17 &  236940.300926 &  3390125.155789 \\ \hline
  f18 &  6522.070266 &  110104.005591 \\ \hline
  f19 &  8.886340 &  32.516194 \\ \hline
  f20 &  13892.000148 &  53412.013643 \\ \hline
  \end{tabular}  
  \label{ion-results}
  \end{table}

  El algoritmo se encuentra claramente lejísimos de todas las metaheurísticas de \textit{differential evolution} [\ref{difev}], y
  por supuesto de las de la competición de CEC2014 [\ref{allresults}].

\section{Propuestas de mejora del algoritmo}
\begin{itemize}
 \item \textbf{Optimización de parámetros}.
 
 A este efecto, hemos probado, para las 5 primeras funciones (debido a la falta de tiempo), los siguientes valores de tamaño
 de población de iones: $\big\{10, 20, 30, 40, 50\big\}$ y los siguientes valores de probabilidad de reinicio: $\big\{0.05, 0.1, 0.15, 0.2\big\}$
 
 Se han obtenido los mejores resultados con un tamaño poblacional de $50$ iones, y una probabilidad de reinicio del $0.1$, que corresponden
 a los resultados presentados en el punto número 3
 
 \item \textbf{Hibridación con búsqueda local}.
 
 Buscamos que el algoritmo tal cual nos proporcione la diversidad y la búsqueda local el factor de convergencia. 
 
 
 Se programó una búsqueda local que no funcionaba comedidamente, que fundamentalmente, realizaba las siguientes operaciones:
 
  \small{\texttt{% Generator: GNU source-highlight, by Lorenzo Bettini, http://www.gnu.org/software/src-highlite
\noindent
\mbox{}applyLocalSearch\textcolor{BrickRed}{(}best$\_$solution\textcolor{BrickRed}{)}\textcolor{Red}{\{} \\
\mbox{}\ \ \ \ current\ \textcolor{BrickRed}{=}\ best$\_$solution \\
\mbox{} \\
\mbox{}\ \ \ \ \textbf{\textcolor{Blue}{for}}\ i\ \textbf{\textcolor{Blue}{in}}\ \textcolor{Red}{\{}\textcolor{Purple}{1}\textcolor{BrickRed}{,}\textcolor{Purple}{2}\textcolor{BrickRed}{,...}evals$\_$ls\textcolor{Red}{\}}\ \textcolor{Red}{\{} \\
\mbox{}\ \ \ \ \ \ \ \ direccion\ \textcolor{BrickRed}{=}\ \textcolor{BrickRed}{[}random\textcolor{BrickRed}{(}\textcolor{Purple}{0}\textcolor{BrickRed}{,}\textcolor{Purple}{1}\textcolor{BrickRed}{)}\ \textbf{\textcolor{Blue}{for}}\ i\ \textbf{\textcolor{Blue}{in}}\ \textcolor{Red}{\{}\textcolor{Purple}{1}\textcolor{BrickRed}{...}dimension\textcolor{Red}{\}}\textcolor{BrickRed}{]} \\
\mbox{}\ \ \ \ \ \ \ \ norma\ \textcolor{BrickRed}{=}\ \textcolor{BrickRed}{$|$$|$}\ direccion\ \textcolor{BrickRed}{$|$$|$} \\
\mbox{} \\
\mbox{}\ \ \ \ \ \ \ \ \textbf{\textcolor{Blue}{for}}\ j\ \textbf{\textcolor{Blue}{in}}\ \textcolor{Red}{\{}\textcolor{Purple}{1}\textcolor{BrickRed}{...}dimension\textcolor{Red}{\}}\ \textcolor{Red}{\{} \\
\mbox{}\ \ \ \ \ \ \ \ \ \ \ \ current\textcolor{BrickRed}{[}j\textcolor{BrickRed}{]}\ \textcolor{BrickRed}{+=}\ random\textcolor{BrickRed}{(}\textcolor{Purple}{0}\textcolor{BrickRed}{,}\ epsilon\textcolor{BrickRed}{)*(}direccion\textcolor{BrickRed}{[}j\textcolor{BrickRed}{]}\ \textcolor{BrickRed}{/}\ norma\textcolor{BrickRed}{);} \\
\mbox{}\ \ \ \ \ \ \ \ \textcolor{Red}{\}} \\
\mbox{} \\
\mbox{}\ \ \ \ \ \ \ \ normalize\textcolor{BrickRed}{(}current\textcolor{BrickRed}{);} \\
\mbox{}\ \ \ \ \ \ \ \ current\textcolor{BrickRed}{.}updateFitness\textcolor{BrickRed}{();} \\
\mbox{} \\
\mbox{}\ \ \ \ \ \ \ \ \textbf{\textcolor{Blue}{if}}\ \textcolor{BrickRed}{(}current\textcolor{BrickRed}{.}getFitness\textcolor{BrickRed}{()}\ \textcolor{BrickRed}{\textless{}=}\ best$\_$solution\textcolor{BrickRed}{.}getFitness\textcolor{BrickRed}{())}\textcolor{Red}{\{} \\
\mbox{}\ \ \ \ \ \ \ \ \ \ \ \ best$\_$solution\ \textcolor{BrickRed}{=}\ current \\
\mbox{}\ \ \ \ \ \ \ \ \textcolor{Red}{\}} \\
\mbox{}\ \ \ \ \ \ \ \ \textbf{\textcolor{Blue}{else}}\textcolor{Red}{\{} \\
\mbox{}\ \ \ \ \ \ \ \ \ \ \ \ current\ \textcolor{BrickRed}{=}\ best$\_$solution \\
\mbox{}\ \ \ \ \ \ \ \ \textcolor{Red}{\}} \\
\mbox{}\ \ \ \ \ \ \ \ i\textcolor{BrickRed}{++;} \\
\mbox{}\ \ \ \ \textcolor{Red}{\}} \\
\mbox{}\textcolor{Red}{\}} \\
\mbox{}
}}
  \normalsize
  
  donde se escogía un \texttt{epsilon = 0.25} y \texttt{evals\_ls = 10*dimension}, es decir, se sumaba un vector de norma euclídea
  menor o igual a epsilon al vector mejor solución, y si encontrábamos mejora, aplicábamos el mismo procedimiento al vector mejorado,
  hasta agotar las $10*D$ iteraciones.
  
 Posteriormente, probamos a hibridar con \textit{SolisWets}, \textit{CMAES} y \textit{Simplex}, búsquedas locales programadas en el software Realea de Daniel Molina distribuido
 bajo licencia GPL. La estrategia para hibridar fue introducir una búsqueda local en fase sólida sobre la mejor solución
 encontrada hasta el momento. Empleamos $10\cdot D$ iteraciones en \textit{Solis Wets}, en \textit{Simplex} y $20\cdot D$ en \textit{CMAES}.
 
 
 Los resultados obtenidos, han sido:
 
 \begin{itemize}
  \item \textbf{Solis Wets}
  \begin{table}[H]	
  \caption{Resultados del \textit{Algoritmo de iones + Solis Wets}}
  \centering
  \begin{tabular}{|l|r|r|||r|r|}
  \hline
  & \textbf{D=10, IMO} & \textbf{D=10, IMO + SW} & \textbf{D=30, IMO} & \textbf{D=30, IMO + SW}\\ \hline
  f1 &  19084554.669857 &  363696.039099 &  78738913.351420 &  455502.264533 \\ \hline
  f2 &  16536.801547 &  1313.954155 &  13697676.211767 &  13127.874215 \\ \hline
  f3 &  21065.221623 &  18688.493110 &  77800.939852 &  71748.162238 \\ \hline
  f4 &  58.764875 &  24.202097 &  246.343748 &  64.979294 \\ \hline
  f5 &  20.170120 &  19.999690 &  20.468675 &  19.999692 \\ \hline
  f6 &  7.539682 &  7.402224 &  34.790689 &  34.666686 \\ \hline
  f7 &  0.861159 &  0.353630 &  1.102618 &  0.011314 \\ \hline
  f8 &  28.934028 &  26.187262 &  129.837295 &  124.369383 \\ \hline
  f9 &  26.123286 &  25.470889 &  145.730468 &  158.078285 \\ \hline
  f10 &  1157.285877 &  948.631408 &  4219.909911 &  3889.044913 \\ \hline
  f11 &  1144.876264 &  1066.564468 &  4881.285910 &  4295.850074 \\ \hline
  f12 &  0.661762 &  0.496767 &  1.304201 &  1.094158 \\ \hline
  f13 &  0.456650 &  0.311175 &  0.440387 &  0.432896 \\ \hline
  f14 &  0.635537 &  0.382150 &  0.260939 &  0.257654 \\ \hline
  f15 &  4.164066 &  3.857350 &  51.414376 &  59.799756 \\ \hline
  f16 &  3.390798 &  3.449191 &  12.366670 &  12.554226 \\ \hline
  f17 &  236940.300926 &  60438.604437 &  3390125.155789 &  46233.218822 \\ \hline
  f18 &  6522.070266 &  7802.026855 &  110104.005591 &  1599.150327 \\ \hline
  f19 &  8.886340 &  7.871310 &  32.516194 &  38.234215 \\ \hline
  f20 &  13892.000148 &  7092.191695 &  53412.013643 &  31560.507434 \\ \hline
  \end{tabular}
  \end{table}

  
  \item \textbf{CMAES}
  \begin{table}[H]	
  \caption{Resultados del \textit{Algoritmo de iones + CMAES}}
  \centering
  \begin{tabular}{|l|r|r|||r|r|}
  \hline
  & \textbf{D=10, IMO} & \textbf{D=10, IMO + CMAES} & \textbf{D=30, IMO} & \textbf{D=30, IMO + CMAES}\\ \hline
  f1 &  19084554.669857 &  7787409.210024 &  78738913.351420 &  603080.766525 \\ \hline
  f2 &  16536.801547 &  1622.048245 &  13697676.211767 &  9668.463875 \\ \hline
  f3 &  21065.221623 &  17145.712244 &  77800.939852 &  80051.795107 \\ \hline
  f4 &  58.764875 &  23.531004 &  246.343748 &  27.503332 \\ \hline
  f5 &  20.170120 &  19.999604 &  20.468675 &  19.999750 \\ \hline
  f6 &  7.539682 &  7.650586 &  34.790689 &  34.910613 \\ \hline
  f7 &  0.861159 &  0.783753 &  1.102618 &  0.007092 \\ \hline
  f8 &  28.934028 &  27.460799 &  129.837295 &  122.737655 \\ \hline
  f9 &  26.123286 &  22.127844 &  145.730468 &  151.949401 \\ \hline
  f10 &  1157.285877 &  1087.485644 &  4219.909911 &  3899.973468 \\ \hline
  f11 &  1144.876264 &  1070.067969 &  4881.285910 &  4053.327961 \\ \hline
  f12 &  0.661762 &  0.568838 &  1.304201 &  0.787965 \\ \hline
  f13 &  0.456650 &  0.310108 &  0.440387 &  0.387936 \\ \hline
  f14 &  0.635537 &  0.454480 &  0.260939 &  0.245177 \\ \hline
  f15 &  4.164066 &  3.668623 &  51.414376 &  57.776390 \\ \hline
  f16 &  3.390798 &  3.375496 &  12.366670 &  12.408639 \\ \hline
  f17 &  236940.300926 &  191239.202539 &  3390125.155789 &  53082.455802 \\ \hline
  f18 &  6522.070266 &  7052.486699 &  110104.005591 &  2039.587266 \\ \hline
  f19 &  8.886340 &  7.586558 &  32.516194 &  40.452787 \\ \hline
  f20 &  13892.000148 &  9850.321392 &  53412.013643 &  34683.954916 \\ \hline
  \end{tabular}
  \end{table}
  
  
  \item \textbf{Simplex}
  \begin{table}[H]	
  \caption{Resultados del \textit{Algoritmo de iones + Simplex}}
  \centering
  \begin{tabular}{|l|r|r|||r|r|}
  \hline
  & \textbf{D=10, IMO} & \textbf{D=10, IMO + Simplex} & \textbf{D=30, IMO} & \textbf{D=30, IMO + Simplex}\\ \hline
  f1 &  19084554.669857 &  1504465.672234 &  78738913.351420 &  4346086.095596 \\ \hline
  f2 &  16536.801547 &  12474.120134 &  13697676.211767 &  2561077.417506 \\ \hline
  f3 &  21065.221623 &  7833.146464 &  77800.939852 &  49242.768978 \\ \hline
  f4 &  58.764875 &  28.881786 &  246.343748 &  89.359609 \\ \hline
  f5 &  20.170120 &  20.074512 &  20.468675 &  20.095649 \\ \hline
  f6 &  7.539682 &  6.171463 &  34.790689 &  30.754850 \\ \hline
  f7 &  0.861159 &  0.571411 &  1.102618 &  0.062850 \\ \hline
  f8 &  28.934028 &  16.717241 &  129.837295 &  67.033718 \\ \hline
  f9 &  26.123286 &  25.795814 &  145.730468 &  150.276753 \\ \hline
  f10 &  1157.285877 &  685.986171 &  4219.909911 &  2185.469491 \\ \hline
  f11 &  1144.876264 &  1030.179614 &  4881.285910 &  3892.796287 \\ \hline
  f12 &  0.661762 &  0.431555 &  1.304201 &  0.501540 \\ \hline
  f13 &  0.456650 &  0.315380 &  0.440387 &  0.466117 \\ \hline
  f14 &  0.635537 &  0.393064 &  0.260939 &  0.264987 \\ \hline
  f15 &  4.164066 &  3.016509 &  51.414376 &  30.881912 \\ \hline
  f16 &  3.390798 &  3.327450 &  12.366670 &  12.234189 \\ \hline
  f17 &  236940.300926 &  209208.552698 &  3390125.155789 &  841071.802922 \\ \hline
  f18 &  6522.070266 &  7622.195964 &  110104.005591 &  1837.062837 \\ \hline
  f19 &  8.886340 &  4.769108 &  32.516194 &  19.797393 \\ \hline
  f20 &  13892.000148 &  5046.809383 &  53412.013643 &  41976.014409 \\ \hline
  \end{tabular}
  \end{table}
  
 \end{itemize}
 
  De las hibridaciones hechas, la que mejor ha funcionado ha sido IMO+Simplex, pero se halla aún muy lejos de resultados
  competitivos.
 \item \textbf{Modelado de fuerzas de atracción y repulsión completas}:
 
 Hemos tratado de modelar todas las fuerzas de atracción-repulsión (no tan solo las de atracción hacia
 el mejor catión o anión, como proponían sus autores). Asimismo, las fuerzas que establecían los autores
 originales se basaban en la distancia en $\mathbb{R}$ de las proyecciones de cada coordenada de las soluciones
 al mejor catión/anión. Se ha intentado modelar estas fuerzas por medio de la calidad del fitness de la función
 a minimizar. 
 
 Explicaremos el algoritmo de movimiento de los iones en el espacio con aniones. Para los cationes, el procedimiento es
 análogo, intercambiando el papel de aniones y cationes en lo que se pasa a explicar a continuación:
 
 Las fuerzas de repulsión entre aniones, se modelan, para un anión $A_i$ dado como:
 
 $$F_{i,j} = \frac{1}{2} \cdot \left\{\frac{1}{1+e^{d_{i,j}}}\right\} \qquad \qquad d_{i,j} = {f(A_i) - f(A_j)}$$ 
 
 Eso implica que si $A_j$ es mucho mejor que $A_i$, apenas se producirá repulsión, mientras que si $A_j$ es mucho peor que $A_i$
 aniones tienen \textit{fit} distintos, la fuerza de repulsión será cercana a $\frac{1}{2}$, actualizándose las componentes de $A_i$
 a:
  $$A_i = A_i + \sum_{j=1, j\neq i}^{j=|Aniones|} rand(0,1)\cdot F_{i,j}\cdot (A_i - A_j)_{i=1\ldots D}$$
  
 Las fuerzas de atracción de los aniones al anión dado $A_i$ se modelan como:
 
 $$F_{i,j} = \frac{1}{2} \cdot \left\{\frac{1}{1+e^{d_{i,j}}}\right\} \qquad \qquad d_{i,j} = {f(C_j) - f(A_i)}$$ 

 De esta forma, la fuerza de atracción es muy alta cuando el catión tiene un \textit{fit} mucho mejor(menor) que el anión, y
 es más pequeña conforme el catión tenga un \textit{fit} mucho más alto que el anión.
 y se actualiza $A_i$ como sigue:
 
 $$A_i = A_i + \sum_{j=1}^{j=|Cationes|} rand(0,1)\cdot F_{i,j}\cdot (C_j - A_i)_{i=1\ldots D}$$
 
 Esta funcionalidad se ha implementado en:
 
 \texttt{void updateLocations(vector<Solution> \&anions, vector<Solution> cations)}
 
 en el fichero \texttt{metaheauristic.cc}. El resto del algoritmo permanece igual que lo descrito hasta el momento, con una búsqueda
 local de tipo \textit{Solis Wets} de $10\cdot D$ iteraciones para la mejor solución encontrada hasta el momento en la fase sólida.
 
 También hemos intentado modelar el mecanismo de reinicio de soluciones (\texttt{void redistribute(vector<Solution> \&ions, vector<Solution> \&bests)}),
 intentando mejorar la diversidad del mismo. LLevamos dos listas, una con los mejores aniones/cationes encontrados hasta el momento
 (cada vez que calculamos el mejor anión o catión). Introducimos también una pequeña mutación con probabilidad de $0.001$
 que hace que una componente de una posible solución se modifique a un número aleatorio entre -100 y 100 con dicha probabilidad.
 El procedimiento seguido en la nueva función de reinicio de soluciones ha sido, fundamentalmente:
 
  \small{\texttt{% Generator: GNU source-highlight, by Lorenzo Bettini, http://www.gnu.org/software/src-highlite
\noindent
\mbox{}\textbf{\textcolor{Blue}{def}}\ redistribute\ \textcolor{BrickRed}{(}ions\textcolor{BrickRed}{,}\ list$\_$bests\textcolor{BrickRed}{)}\textcolor{Red}{\{} \\
\mbox{}\ \ \ \ \textbf{\textcolor{Blue}{for}}\ ion\ \textbf{\textcolor{Blue}{in}}\ ions\textcolor{Red}{\{} \\
\mbox{}\ \ \ \ \ \ \ \ \textbf{\textcolor{Blue}{if}}\ rand\textcolor{BrickRed}{(}\textcolor{Purple}{0}\textcolor{BrickRed}{,}\textcolor{Purple}{1}\textcolor{BrickRed}{)}\ \textcolor{BrickRed}{\textless{}}\ prob$\_$restart \\
\mbox{}\ \ \ \ \ \ \ \ \ \ \ \ ion\textcolor{BrickRed}{[}j\textcolor{BrickRed}{]}\ \textcolor{BrickRed}{=}\ \textcolor{BrickRed}{[}random\textcolor{BrickRed}{(-}\textcolor{Purple}{100}\textcolor{BrickRed}{,}\textcolor{Purple}{100}\textcolor{BrickRed}{)}\ \textbf{\textcolor{Blue}{for}}\ j\ \textbf{\textcolor{Blue}{in}}\ \textcolor{Red}{\{}\textcolor{Purple}{1}\textcolor{BrickRed}{...}length\textcolor{BrickRed}{(}ion\textcolor{BrickRed}{)}\textcolor{Red}{\}}\textcolor{BrickRed}{]} \\
\mbox{}\ \ \ \ \ \ \ \ \textbf{\textcolor{Blue}{else}}\textcolor{Red}{\{} \\
\mbox{}\ \ \ \ \ \ \ \ \ \ \ \ \textbf{\textcolor{Blue}{if}}\ \textcolor{BrickRed}{!}list$\_$bests\textcolor{BrickRed}{.}empty\textcolor{BrickRed}{()}\textcolor{Red}{\{} \\
\mbox{}\ \ \ \ \ \ \ \ \ \ \ \ \ \ \ \ ions\ \textcolor{BrickRed}{=}\ lists$\_$bests\textcolor{BrickRed}{.}pop\textcolor{BrickRed}{()} \\
\mbox{}\ \ \ \ \ \ \ \ \ \ \ \ \textcolor{Red}{\}} \\
\mbox{}\ \ \ \ \ \ \ \ \ \ \ \ \textbf{\textcolor{Blue}{else}}\textcolor{Red}{\{} \\
\mbox{}\ \ \ \ \ \ \ \ \ \ \ \ \ \ \ \ ion\textcolor{BrickRed}{[}j\textcolor{BrickRed}{]}\ \textcolor{BrickRed}{=}\ \textcolor{BrickRed}{[}random\textcolor{BrickRed}{(-}\textcolor{Purple}{100}\textcolor{BrickRed}{,}\textcolor{Purple}{100}\textcolor{BrickRed}{)}\ \textbf{\textcolor{Blue}{for}}\ j\ \textbf{\textcolor{Blue}{in}}\ \textcolor{Red}{\{}\textcolor{Purple}{1}\textcolor{BrickRed}{...}length\textcolor{BrickRed}{(}ion\textcolor{BrickRed}{)}\textcolor{Red}{\}}\textcolor{BrickRed}{]} \\
\mbox{}\ \ \ \ \ \ \ \ \ \ \ \ \textcolor{Red}{\}} \\
\mbox{}\ \ \ \ \ \ \ \ \ \ \ \ \textbf{\textcolor{Blue}{for}}\ j\ \textbf{\textcolor{Blue}{in}}\ \textcolor{Red}{\{}\textcolor{Purple}{1}\textcolor{BrickRed}{...}length\textcolor{BrickRed}{(}ion\textcolor{BrickRed}{)}\textcolor{Red}{\}\{} \\
\mbox{}\ \ \ \ \ \ \ \ \ \ \ \ \ \ \ \ \textbf{\textcolor{Blue}{if}}\textcolor{BrickRed}{(}rand\textcolor{BrickRed}{(}\textcolor{Purple}{0}\textcolor{BrickRed}{,}\textcolor{Purple}{1}\textcolor{BrickRed}{)}\ \textcolor{BrickRed}{\textless{}}\ prob$\_$mutation\textcolor{BrickRed}{)} \\
\mbox{}\ \ \ \ \ \ \ \ \ \ \ \ \ \ \ \ \ \ \ \ ion\textcolor{BrickRed}{[}j\textcolor{BrickRed}{]}\ \textcolor{BrickRed}{=}\ random\textcolor{BrickRed}{(-}\textcolor{Purple}{100}\textcolor{BrickRed}{,}\textcolor{Purple}{100}\textcolor{BrickRed}{)} \\
\mbox{}\ \ \ \ \ \ \ \ \ \ \ \ \textcolor{Red}{\}} \\
\mbox{}\ \ \ \ \ \ \ \ \textcolor{Red}{\}} \\
\mbox{}\ \ \ \ \textcolor{Red}{\}} \\
\mbox{}\textcolor{Red}{\}} \\
\mbox{} \\
\mbox{}redistribute\textcolor{BrickRed}{(}anions\textcolor{BrickRed}{,}\ best$\_$cations\textcolor{BrickRed}{)} \\
\mbox{}redistribute\textcolor{BrickRed}{(}cations\textcolor{BrickRed}{,}\ best$\_$anions\textcolor{BrickRed}{)} \\
\mbox{}
}}
  \normalsize 
 
 %Asimismo, se establece como condición para salir de la fase líquida no haber actualizado la mejor solución\ldots
 \begin{table}[H]	
  \caption{Resultados del \textit{Algoritmo de iones mejorado + Simplex}}
  \centering
  \begin{tabular}{|l|r|r|||r|r|}
  \hline
  & \textbf{D=10, IMO} & \textbf{D=10, IMO mejorado} & \textbf{D=30, IMO} & \textbf{D=30, IMO mejorado}\\ \hline
  f1 &  19084554.669857 &  26329.979483 &  78738913.351420 &  230907.346453 \\ \hline
  f2 &  16536.801547 &  3721.020604 &  13697676.211767 &  9338.377990 \\ \hline
  f3 &  21065.221623 &  18588.818078 &  77800.939852 &  73035.251038 \\ \hline
  f4 &  58.764875 &  26.436828 &  246.343748 &  19.987042 \\ \hline
  f5 &  20.170120 &  20.121005 &  20.468675 &  20.274226 \\ \hline
  f6 &  7.539682 &  8.503934 &  34.790689 &  39.073894 \\ \hline
  f7 &  0.861159 &  1.562235 &  1.102618 &  0.009251 \\ \hline
  f8 &  28.934028 &  37.319324 &  129.837295 &  293.089630 \\ \hline
  f9 &  26.123286 &  49.547829 &  145.730468 &  330.339667 \\ \hline
  f10 &  1157.285877 &  970.595306 &  4219.909911 &  5467.684508 \\ \hline
  f11 &  1144.876264 &  1054.685361 &  4881.285910 &  5191.974227 \\ \hline
  f12 &  0.661762 &  0.739751 &  1.304201 &  1.809880 \\ \hline
  f13 &  0.456650 &  0.484668 &  0.440387 &  0.498159 \\ \hline
  f14 &  0.635537 &  0.490654 &  0.260939 &  0.356620 \\ \hline
  f15 &  4.164066 &  26.022467 &  51.414376 &  182.177060 \\ \hline
  f16 &  3.390798 &  3.541966 &  12.366670 &  12.956010 \\ \hline
  f17 &  236940.300926 &  5905.531778 &  3390125.155789 &  16890.938744 \\ \hline
  f18 &  6522.070266 &  9420.863246 &  110104.005591 &  8978.058073 \\ \hline
  f19 &  8.886340 &  8.106570 &  32.516194 &  105.267242 \\ \hline
  f20 &  13892.000148 &  4783.046252 &  53412.013643 &  31203.001882 \\ \hline
  \end{tabular}
  \end{table}
 
 
 Como podemos observar, hemos conseguido mejorar mucho los resultados más negativos de la anterior versión del algoritmo, pero
 en contraposición, hemos empeorado un poco los mejores resultados.
 
 \item Algoritmo mejorado v2 + Búsqueda local con nuevo equilibrio diversidad-convergencia
 
 Introducimos más variaciones en el algoritmo con objetivo de mejorar su diversidad, y damos más potencia a CMAES (búsqueda
 local más potente de las que disponemos, pero que más evaluaciones consume). En cada fase sólida, ejecutamos CMAES con tope
 de evaluaciones $D \cdot 10000$. 
 
 Algunas de las variaciones introducidas, respecto a lo descrito anteriormente, han sido:
 \begin{itemize}
  \item Intercambio de cationes y de aniones al comenzar cada fase líquida
  \item Presencia sólo de fuerzas atractivas, no repulsivas. Las fuerzas de atracción son las descritas en la variación anterior.
  Para aniones: 
  
     $$F_{i,j} = \frac{1}{2} \cdot \left\{\frac{1}{1+e^{d_{i,j}}}\right\} \qquad \qquad d_{i,j} = {f(C_j) - f(A_i)}$$ 
     
  Para cationes, análogo.
  
  \item Los aniones se actualizan con la lista de mejores aniones hasta el momento, los cationes con la lista de mejores cationes
  hasta el momento, y no al revés como en la variación anterior, ya que es indistinto, si cambiamos a cada paso el papel
  de los aniones y el de los cationes.
  
  \item La fase líquida se para tras un tope de evaluaciones. Hemos seleccionado 1000 evaluaciones como tope, que parece
  estar dando buenos resultados, tras probar otros valores.
 \end{itemize}
 
  Obtenemos unos resultados sorprendentemente buenos, en comparación a los resultados del
 algoritmo de iones desde los que partiamos. 
 
  \begin{table}[H]	
  \caption{Resultados del \textit{Algoritmo de iones mejorado v2 + CMAES}}
  \centering
  \begin{tabular}{|l|l|l|}
  \hline
  & \textbf{D=10, IMO mejorado} & \textbf{D=30, IMO mejorado} \\ \hline
  f1 &  0.000000 &  4355.835454 \\ \hline
  f2 &  0.000000 &  0.000000 \\ \hline
  f3 &  0.000000 &  0.000001 \\ \hline
  f4 &  0.318926 &  7.516527 \\ \hline
  f5 &  20.384630 &  20.871326 \\ \hline
  f6 &  0.997974 &  5.864659 \\ \hline
  f7 &  0.008965 &  0.001972 \\ \hline
  f8 &  13.889611 &  48.354929 \\ \hline
  f9 &  12.854858 &  54.205256 \\ \hline
  f10 &  853.542288 &  3346.864204 \\ \hline
  f11 &  968.525040 &  3145.585325 \\ \hline
  f12 &  0.399067 &  1.402973 \\ \hline
  f13 &  0.114891 &  0.247605 \\ \hline
  f14 &  0.454001 &  0.434418 \\ \hline
  f15 &  1.161012 &  4.006343 \\ \hline
  f16 &  3.920165 &  13.300889 \\ \hline
  f17 &  591.667610 &  1780.139778 \\ \hline
  f18 &  62.436072 &  348.075348 \\ \hline
  f19 &  3.017356 &  14.741923 \\ \hline
  f20 &  68.095386 &  458.587897 \\ \hline
  \end{tabular}
  \label{mejv2}
  \end{table}
  
  Nos replanteamos si con este nuevo equilibrio diversidad-convergencia podemos mejorar los resultados de la versión de los autores.
  Efectivamete, nos encontramos que con $D\cdot 1000$ evaluaciones de esta función, los resultados que obtenemos son satisfactoriamente
  mejores:
  
  \begin{table}[H]	
  \caption{Resultados del \textit{Algoritmo de iones mejorado v2 + CMAES}}
  \centering
  \begin{tabular}{|l|l|l|}
  \hline
  & \textbf{D=10, IMO mejorado} & \textbf{D=30, IMO mejorado} \\ \hline
  f1 &  0.000000 &  153.994848 \\ \hline
  f2 &  0.000000 &  0.000000 \\ \hline
  f3 &  0.000000 &  0.000000 \\ \hline
  f4 &  0.000000 &  0.147572 \\ \hline
  f5 &  20.001247 &  20.622110 \\ \hline
  f6 &  0.000002 &  0.357917 \\ \hline
  f7 &  0.000000 &  0.000000 \\ \hline
  f8 &  4.576812 &  27.580246 \\ \hline
  f9 &  4.576811 &  27.699639 \\ \hline
  f10 &  283.681734 &  1693.500771 \\ \hline
  f11 &  299.088276 &  2233.018358 \\ \hline
  f12 &  0.029925 &  0.019306 \\ \hline
  f13 &  0.044096 &  0.169150 \\ \hline
  f14 &  0.275749 &  0.281541 \\ \hline
  f15 &  0.566455 &  2.281242 \\ \hline
  f16 &  3.267757 &  13.005128 \\ \hline
  f17 &  252.999183 &  1340.708149 \\ \hline
  f18 &  44.665859 &  286.135882 \\ \hline
  f19 &  1.593592 &  11.621429 \\ \hline
  f20 &  49.247290 &  505.252065 \\ \hline
  \end{tabular}
  \end{table}
  
  Estos constituyen los mejores resultados que hemos obtenido con el algoritmo.
  
\end{itemize}



\section{Empleo del software programado}

Para compilar el software proporcionado, basta hacer, en el directorio \texttt{src}, lo siguiente:

\begin{verbatim}
    Cmake .
    make
    ./main
\end{verbatim}

Cuando ejecutamos el programa, obtenemos para dimensión 10 y 30, la evaluación que producen el algoritmo de iones.
Si se quiere modificar algún parámetro, debe hacerse en el fichero \texttt{./src/aux.cpp}. 

\begin{itemize}
 \item \texttt{num\_ejecuciones}. Fijado a 25.
 \item \texttt{population\_size}. Tamaño de la población total de iones en cada iteración. La mitad serán aniones y la otra
 mitad cationes.
 \item \texttt{lbound} y \texttt{ubound}. Son los límites del espacio $[lbound, ubound]^D$.
 \item \texttt{prob\_restart}. Probabilidad de reinicio de aniones y cationes. Por defecto a 0.1.
 \item \texttt{type\_ls}. Puede ser ``sw``, ``cmaes`` o ''simplex``. Tipo de búsqueda local a emplear.
\end{itemize}

Si se ha modificado algún parámetro, habrá que hacer:
\begin{verbatim}
    make
    ./main
\end{verbatim}

El \texttt{./main} está configurado para obtener los resultados con el mejor algoritmo que hemos encontrado hasta el momento,
la versión de los autores hibridada con CMAES a $1000\cdot D$ evaluaciones en fase sólida.

Si cambiamos en \texttt{main} \texttt{ionAlgorithm()} por \texttt{ionAlgorithm\_v2()} obtendremos los resultados de \ref{mejv2}



\section{Conclusiones aprendidas}

\section{Anexo: Resultados para comparar}

  \begin{table}[H]	
  \caption{Resultados del \textit{Resultados de Differential Evolution}}
  \centering
  \begin{tabular}{|l|r|r|r|r|}
  \hline
     & \textbf{D=10, DEBin} & \textbf{D=10, DExp} & \textbf{D=30, DEBin} & \textbf{D=30, DExp}\\ \hline
  f1 & 0.00E+00 & 0.00E+00 & 8.88E+04 & 2.95E+05 \\ \hline
  f2 & 0.00E+00 & 0.00E+00 & 2.00E+02 & 2.00E+02 \\ \hline
  f3 & 0.00E+00 & 0.00E+00 & 3.00E+02 & 3.00E+02 \\ \hline
  f4 & 2.16E+01 & 2.19E+01 & 3.85E+02 & 4.11E+02 \\ \hline
  f5 & 2.02E+01 & 2.01E+01 & 5.01E+02 & 5.00E+02 \\ \hline
  f6 & 5.84E-01 & 1.84E-01 & 6.03E+02 & 6.17E+02 \\ \hline
  f7 & 3.66E-02 & 3.58E-02 & 7.00E+02 & 7.00E+02 \\ \hline
  f8 & 4.35E+00 & 3.98E-02 & 8.10E+02 & 8.00E+02 \\ \hline
  f9 & 1.20E+01 & 8.79E+00 & 1.02E+03 & 9.73E+02 \\ \hline
  f10 & 5.20E+01 & 1.15E+01 & 1.43E+03 & 9.93E+02 \\ \hline
  f11 & 4.44E+02 & 5.55E+02 & 6.76E+03 & 4.31E+03 \\ \hline
  f12 & 4.26E-01 & 4.39E-01 & 1.20E+03 & 1.20E+03 \\ \hline
  f13 & 1.14E-01 & 1.12E-01 & 1.30E+03 & 1.30E+03 \\ \hline
  f14 & 1.76E-01 & 1.63E-01 & 1.40E+03 & 1.40E+03 \\ \hline
  f15 & 1.77E+00 & 1.24E+00 & 1.51E+03 & 1.51E+03 \\ \hline
  f16 & 2.34E+00 & 2.35E+00 & 1.61E+03 & 1.61E+03 \\ \hline
  f17 & 1.64E+01 & 8.49E+00 & 4.95E+03 & 3.93E+03 \\ \hline
  f18 & 5.40E-01 & 5.54E-01 & 1.82E+03 & 1.84E+03 \\ \hline
  f19 & 3.11E-01 & 4.53E-01 & 1.90E+03 & 1.91E+03 \\ \hline
  f20 & 2.04E-01 & 6.60E-02 & 2.01E+03 & 2.03E+03 \\ \hline
  \end{tabular}
  \label{difev}
  \end{table}

  donde \textbf{DEBin} es un \textit{differential evolution} con operador de mutación \texttt{rand/1/bin}; y
  \textbf{DExp} es un \textit{differential evolution} con operador de mutación \texttt{rand/1/exp}
  
  \begin{table}[H]
  \caption{Resultados de CEC2014}
   \begin{adjustbox}{width=1.15\textwidth}
  \centering
  \begin{tabular}{|c|r|r|r|r|r|r|r|r|r|r|}
  \hline
  \multicolumn{1}{|l|}{} & \multicolumn{1}{c|}{\textbf{CoDE}} & \multicolumn{1}{c|}{\textbf{D-SHADE}} & \multicolumn{1}{c|}{\textbf{EPSDE}} & \multicolumn{1}{c|}{\textbf{JADE}} & \multicolumn{1}{c|}{\textbf{L-SHADE}} & \multicolumn{1}{c|}{\textbf{NBIPOP-aCMA-ES}} & \multicolumn{1}{c|}{\textbf{SHADE11}} & \multicolumn{1}{c|}{\textbf{SaDE}} & \multicolumn{1}{c|}{\textbf{dynNP-jDE}} & \multicolumn{1}{c|}{\textbf{iCMAES-ILS}} \\ \hline
  \textbf{f1} & 0 & 0 & 0 & 0 & 0 & 0 & 0 & 2.5523185874 & 2.16932408E-007 & 0 \\ \hline
  \textbf{f2} & 0 & 0 & 0 & 0 & 0 & 0 & 0 & 0 & 0 & 0 \\ \hline
  \textbf{f3} & 0 & 0 & 0 & 0.0061659368 & 0 & 0 & 0 & 0 & 0 & 0 \\ \hline
  \textbf{f4} & 10.4826142934 & 30.7734858177 & 0 & 27.6186789086 & 29.4095534696 & 2.8198864768 & 29.4945614566 & 18.0850398784 & 3.3229206052 & 14.3852921475 \\ \hline
  \textbf{f5} & 18.4489901182 & 17.7268744007 & 20.0499565405 & 17.2676724191 & 14.1455983633 & 18.0702454036 & 18.0061827299 & 15.7840675315 & 15.9500415847 & 14.6543978013 \\ \hline
  \textbf{f6} & 1.652545E-006 & 0 & 3.0415621869 & 0.1755220197 & 0.0175400641 & 0.3305270985 & 0 & 0 & 0 & 0 \\ \hline
  \textbf{f7} & 0.0375906649 & 0.0053139273 & 0.0175738115 & 0.0118615958 & 0.0030429237 & 0 & 0.0097818861 & 0.0072428745 & 0.0049691893 & 0 \\ \hline
  \textbf{f8} & 0 & 0 & 0 & 0 & 0 & 3.6972501813 & 0 & 0 & 0 & 0.2536577111 \\ \hline
  \textbf{f9} & 3.8822907289 & 3.0826834295 & 3.6887332129 & 3.5070887497 & 2.344597678 & 0.326222014 & 3.1407481567 & 3.5837731043 & 3.8578530972 & 0.0975487633 \\ \hline
  \textbf{f10} & 0.0355133096 & 0.0489838753 & 0.0440854878 & 0.0061229844 & 0.0085721782 & 91.6179225313 & 0.0122459688 & 0.0195935501 & 0.0024491938 & 122.0401124998 \\ \hline
  \textbf{f11} & 75.97028827 & 54.9344512263 & 323.1316686571 & 83.6964918713 & 32.055826349 & 116.8327825644 & 63.1801993237 & 196.4226016765 & 136.2141998624 & 8.5850809801 \\ \hline
  \textbf{f12} & 0.0433839876 & 0.0529084306 & 0.3210496968 & 0.2501378312 & 0.0681671385 & 0.0100833433 & 0.1364084116 & 0.4350190349 & 0.3111217612 & 0.0650093797 \\ \hline
  \textbf{f13} & 0.0797987518 & 0.0489190311 & 0.1223680668 & 0.0839677932 & 0.0515619667 & 0.010894762 & 0.073996527 & 0.1252053565 & 0.1189658945 & 0.0091148563 \\ \hline
  \textbf{f14} & 0.1071541378 & 0.0900969438 & 0.1363222239 & 0.1105412155 & 0.081361704 & 0.2824303933 & 0.1055165522 & 0.1857701461 & 0.1352218727 & 0.1545591085 \\ \hline
  \textbf{f15} & 0.6523180198 & 0.4027667033 & 0.7538668012 & 0.5782542708 & 0.3660991591 & 0.5467188825 & 0.5051689656 & 0.7903287228 & 0.7815061018 & 0.7233311235 \\ \hline
  \textbf{f16} & 1.1291916406 & 1.3390288512 & 2.5412985701 & 1.6508403475 & 1.2407967834 & 2.5295163003 & 1.5571401888 & 1.9669671832 & 1.5944840344 & 1.9070502497 \\ \hline
  \textbf{f17} & 2.6621315275 & 3.3813450543 & 53.2895801695 & 30.9109611068 & 0.9766637611 & 38.8964541152 & 1.5576914381 & 28.3335977004 & 2.622821365 & 21.0348886039 \\ \hline
  \textbf{f18} & 0.430562228 & 0.4749765157 & 1.1971258083 & 0.238780539 & 0.2440927767 & 3.5766693718 & 0.2369542912 & 1.6451056175 & 0.4409595631 & 0.5259199747 \\ \hline
  \textbf{f19} & 0.0744713059 & 0.2050414984 & 1.4319443123 & 0.2549178006 & 0.0773000324 & 0.8276950139 & 0.1916908903 & 0.0668896409 & 0.1218315053 & 0.7076957291 \\ \hline
  \textbf{f20} & 0.0239118917 & 0.2733571191 & 0.1650332637 & 0.3240730933 & 0.1848826079 & 1.3167908894 & 0.2433489516 & 0.1076009171 & 0.0414788094 & 0.8040941367 \\ \hline
  \end{tabular}
  \end{adjustbox}
  \label{allresults}
  \end{table}
  
\end{document}